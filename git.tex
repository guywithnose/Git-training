\documentclass[14pt]{beamer}
\usepackage[latin1]{inputenc}
\usepackage{graphics}
\usepackage{hyperref}
\usetheme{Warsaw}
\setbeamertemplate{navigation symbols}{}
\title{The Basics of Git}
\begin{document}
\begin{frame}
\titlepage
\end{frame}

\begin{frame}{What is Git?}
	According to the documentation:\newline
	"git" can mean anything, depending on your mood.
	\begin{enumerate}
		\small
		\item random three-letter combination that is pronounceable, and not
		   actually used by any common UNIX command.  The fact that it is a
		   mispronunciation of "get" may or may not be relevant.
		\item stupid. contemptible and despicable. simple. Take your pick from the
		   dictionary of slang.
		\item "global information tracker": you're in a good mood, and it actually
		   works for you. Angels sing, and a light suddenly fills the room.
		\item "godd*mn idiotic truckload of sh*t": when it breaks
	\end{enumerate}
\end{frame}

\begin{frame}{What is Git?}
	Git is a fast, scalable, distributed revision control system with an
	unusually rich command set that provides both high-level operations
	and full access to internals.
\end{frame}

\begin{frame}{What is Git?}
	\begin{enumerate}
		\small
		\item Git is flexible for many different workflows.
		\item Git is non-linear.
		\item Git is offline and therefore much faster.
		\item Git is distributed.  This means it allows for many
			developers to work independenty, yet collaboratively.
		\item Git allows a deveoper to seemlessly switch between features(branches).
		\item Git allows for disposable experimentation.
	\end{enumerate}
\end{frame}


\begin{frame}{Getting started}
	\begin{enumerate}
		\small
		\item Install git. (mSysGit on Windows)\newline
		\href{http://msysgit.googlecode.com/files/Git-1.7.11-preview20120710.exe}{http://msysgit.googlecode.com/files/Git...}
		\href{http://goo.gl/uXDLa}{http://goo.gl/uXDLa}
		\item Get a github account
		\item Set up SSH keys
		\item Forking a repository\newline
		\href{https://github.com/guywithnose/Git-training}{https://github.com/guywithnose/Git-training}
		\href{http://goo.gl/EDor7}{http://goo.gl/EDor7}
		\item Clone your fork
		\item Commit
		\item Push
		\item Merge/Pull Request
	\end{enumerate}
\end{frame}


\begin{frame}{Basic commands}
	add
	reset
	status
	commit
	log
	diff
	show
\end{frame}

\begin{frame}{Staging area}
	image
           add            commit
Working Dir -> Staging Area -> HEAD(Commited)
            <-              <-
          reset        reset --soft

Status lets you know what files are in what state.
\end{frame}

\begin{frame}{What is a commit}
	image from pro git
\end{frame}

\begin{frame}{Logging the history}
	Unlike SVN git is not linear.  A commit can have two parents.
\end{frame}

\end{document}